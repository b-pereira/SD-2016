
\chapter*{Conclusão}
\addcontentsline{toc}{chapter}{Conclusão}
\refstepcounter{chapter}

Terminada a fase de desenvolvimento do projeto, depois de feita uma minuciosa análise de requisitos e avaliação do problema, chegamos a conclusões (apresentadas ao longo do corpo deste relatório) que nos deixaram satisfeitos. Pensamos ter sido capazes de apresentar explicações detalhadas no sentido em que toda a informação essencial para entender implementação da aplicação em questão e o seu funcionamento se encontra presente sem nenhuma exceção, mas ao mesmo tempo bastante simples na maneira como será possível para alguém com pouco conhecimento de o que é Java, a linguagem usada neste projeto, consiga ainda assim acompanhar e entender o que está a ser dito.

Tendo isto em conta temos também a noção que nunca nenhum projeto se pode dar por realmente terminado ou por realmente perfeito, há sempre a possibilidade de melhorar algo e esse fato não deve nunca ser ignorado, por mais experiência que venhamos a ter, existe sempre um limite para quão boas as nossas primeiras impressões poderão ser. Desta forma estariamos preparados, caso tivessemos sido contratados por uma empresa, para no futuro a pedido da mesma modificar algo que aqui poderá ter sido afirmado e/ou concluído. Isto é, o nosso projeto foi elaborado com a ideia de criar algo que permita uma fácil expansão tanto das funcionalidades já existentes como de possíveis outras totalmente novas. Isto é essencial em qualquer projeto, pois sabemos que no mundo real o empregador nem sempre sabe de raiz tudo aquilo que realmente quer implementar por isso torna-se natural que, num momento mais avançado da implementação, liste novas capacidades que quer ver presentes. Um projeto mal pensado para expansões e modificações poderia ter de ser refeito de raiz para não dar problemas ou graus baixos de eficiência, no entanto tivemos a ideia presente de tentar criar algo que podesse minimizar todas essas possíveis perdas que poderiam acontecer se este fosse um projeto no mundo real e não um trabalho académico protegido pela redoma que acaba sempre por ser uma universidade.

Esta facilidade de expansão verifica-se, por exemplo, na maneira como se interpretam as mensagens recebidas pela thread de cada utilizador. Na Classe Cliente vemos que adicionar novas mensagens para interpretar se pode fazer apenas com duas linhas sem ter que apagar nada.

Da mesma maneira podemos observar que o código está estruturado e organizado em packages de forma a reduzir ao máximo repetição de código. Métodos estão também divididos em vários de modo a serem o mais pequeno e concisos possível, desta forma será fácil para alguém que não quem programou perceba como a aplicação funciona.   






