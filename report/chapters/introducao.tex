\chapter*{Introdução} 
\addcontentsline{toc}{chapter}{Introdução} 

No presente ano letivo (2015/2016) foi proposto aos alunos da unidade curricular de Sistemas Distribuídos que desenvolvessem uma aplicação, com recurso à linguagem de programação Java, que emulasse o funcionamento de um serviço de táxis recorrendo a sockets TCP/IP e a programação concorrente.

O nosso grupo (Grupo 88) tomou a iniciativa de tentar fazer uma aplicação estruturada e bem pensada de modo a que apenas olhando para o código se consiga facilmente entender o que se pretendeu com cada método e cada estratégia. Tivemos também em conta a dificuldade e a, por vezes pouco intuitiva, natureza da programação concorrente, tudo isso levou-nos a querer esquematizar bem o projeto antes de realmente o começarmos a programar, porque entendemos que não seria eficiente começar de imediato a "escrever qualquer coisa" apenas para no futuro apagar e fazer de novo. Para isso desenvolvemos este relatório simples onde definimos e explicamos a estratégia para realizar cada um dos objetivos a que nos propusemos.  

%Será discutido no capítulo %\ref{cap:devel} , na pagina %\pageref{cap:devel} os detalhes sobre o
%desenvolvimento da aplicação\\[1cm]








