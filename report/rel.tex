\documentclass[pdftex,12pt,a4paper]{report}

\usepackage[pdftex]{graphicx}
\usepackage{float}
\usepackage{fancyvrb}
\fvset{xleftmargin=2em}

\usepackage{pgfplots}
\pgfplotsset{width=10cm,compat=1.9}
\usepackage{tikzscale}
\usepackage{pgfplotstable}
\usepackage{booktabs}
\usepackage[font=small,labelfont=bf,tableposition=top]{caption}

\usepackage[utf8]{inputenc} % isto é um comentário
\usepackage[portuges]{babel}
\usepackage[T1]{fontenc}
\usepackage{times}
%\usepackage{lmodern}
\usepackage[obeyspaces,spaces]{url}
\usepackage[left=25mm,right=25mm,top=25mm,bottom=25mm]{geometry}
\usepackage{titlesec}
\usepackage{mathtools}
\usepackage{amsfonts}

%identa 1º paragrafo de capitulos e secções
\usepackage{indentfirst}



\usepackage[]{hyperref}
\hypersetup{
bookmarksnumbered=true,     
bookmarksopen=true,         
bookmarksopenlevel=1,       
colorlinks=true,            
pdfstartview=Fit,           
pdfpagemode=UseOutlines, % this is the option you were lookin for
pdfpagelayout=TwoPageRight
		}
%\hypersetup{pdftex,colorlinks=true,allcolors=blue}
%\usepackage{hypcap}

\newcommand{\HRule}{\rule{\linewidth}{0.5mm}}
\titleformat{\chapter}{\normalfont\huge}{\thechapter.}{20pt}{\huge}


\begin{document}

\begin{titlepage}


\begin{minipage}{0.3\textwidth}
\begin{flushleft} 
\includegraphics[width=1.1\textwidth]{./report/logo.png}
\end{flushleft}
\end{minipage}
\hfill
\begin{minipage}{0.6\textwidth}
\begin{flushright} 

\fontfamily{pag}\selectfont{\large \textsc{Departamento de Informática}\\[0.1cm]
\large \bfseries Mestrado Integrado em Engenharia Informática \\ [0.1cm]
\large \bfseries \textit{Sistemas Distribuídos}\\[4mm]
}
\noindent\rule{\textwidth}{0.7mm}
\end{flushright}
\end{minipage}\\[1cm]


\vspace{3cm}


\begin{center}

\fontfamily{pag}\selectfont{\textsc{\Huge Gestor de Deslocações}\\[1cm]


{\large \bfseries \emph{Implementação de uma aplicação distribuída para gestão
de um serviço de taxis} \\[1cm] }


{\textsc{\Large Grupo 88}}

\vspace{3cm}

\begin{minipage}{0.4\textwidth}
	\begin{flushleft} 
		\large Bruno Pereira\\
           Aluno nº 72628 
	\end{flushleft}

\end{minipage}
\begin{minipage}{0.4\textwidth}
	\begin{flushright} 
		\large Daniel Malhadas\\
           Aluno nº 72293
	\end{flushright}
\end{minipage}

\vfill
\begin{minipage}{0.4\textwidth}
	\begin{center}
		
\large Alexandre Silva\\
       Aluno nº 72502
	\end{center}

\end{minipage}



\vfill

\emph{\large Braga, {\large \today}}
}
\end{center}

\end{titlepage}

\begin{abstract}
Projeto para a disciplina de Sistemas Distribuídos 2015/2016.
Consiste num serviço de Táxis informatizado onde se inscrevem passageiros e taxistas para, de diferentes formas, interagirem entre si.
\end{abstract}

\tableofcontents

\chapter*{Introdução} 
\addcontentsline{toc}{chapter}{Introdução} 

%Será discutido no capítulo %\ref{cap:devel} , na pagina %\pageref{cap:devel} os detalhes sobre o
%desenvolvimento da aplicação\\[1cm]









\chapter{Aplicação de gestão de serviços de \emph{taxi}}

\section{Análise do Problema}

%1. Hipótese / Objetivo. Em primeiro lugar, referir o que se está a tentar descobrir, ou tentar fazer, da
%froma mais clara possível



\section{Implementação}




\section{Testes}


\section{Resultados}

\section{Discussão Resultados}






\chapter*{Conclusão}
\addcontentsline{toc}{chapter}{Conclusão}
\refstepcounter{chapter}









\end {document}


